\section{Introduction}
This section will provide the reader with a general understanding of what a network protocol is, and the issues involved in implementing such protocols.
\\\\
Often a protocol can be both complicated in its concept as it is hard in implementing. Protocols have a strictly defined control flow that specify the response to various types of messages. Protocols often define rules that these messages must obey in order for it to be a valid message. One such rule could for example be that a message sent contains a value that is a prime number. A protocol will usually have an equal amount of message types as they have rules. The resulting implementation of all of these messages and rules can make a code base difficult to understand and obfuscate the protocols control flow in the implementation.

In this paper we have created a domain specific language that will help us in solving this problem. This DSL provides us with a foundation for defining network protocols where the control flow is clear, presise and syntactically easy to understand. 

\subsection{Issues with Implementation}
The perhaps greatest problem faced when a protocol does not conform to its rules, is that sensitive information might be leaked. In recent years, two such failures have occurred and received large media coverage. Namely Heartbleed \cite{durumeric2014matter} and Apples Goto bug \cite{bland2014finding}. Failure to correctly implement a security protocol may  cause irreparable damage on a companies trustworthiness and reputation.
\\\\
Some of the key issues involved with implementing a network protocol are
\begin{enumerate}
  \item Defining the rules/model of a protocol
  \item Ensuring we abide by those rules
  \item Matching the complexity of the protocol with its implementation
\end{enumerate}

\subsection{Why We Need Network Protocols}
The goal of a network protocol is to allow computers to have a standardised way of communicating with each other. This protocol gives us a shared vocabulary as it contains the definitions for the various aspects contained within the protocol. Without protocols, computers would not be able to communicate. Implementing a simple ``Hello World'' web server would be impossible. While one type of web browser may request the data as ``GET /index.html'', another might request it as ``FETCH /index.html''. The truth is, however, these request would never actually reach the web server. Without any form of protocol we can't expect to send a request over the internet. Specifically, without the TCP/IP protocol \cite{clark1988design}, we would have no way of sending requests/packets over the web.

\subsubsection{Security}
As stated above, using protocols allows for machines to communicate. We can also add security protocols to the mix and ensure that only the intended parties are able to decipher our secure messages. Without this security, we would not be able to use any form of secure online authentication.

\subsubsection{Power in numbers}
A key benefit of using a popular network protocol is that you are using a protocol that others have tested and implemented before. If you encounter problems you can ask for help and use a common vocabulary when doing so. Academia will often set out to test if protocol implementations are as secure we assume. If they discover any defects, they will publish information regarding the security threat. This will allow those who use the protocol to update their implementation and remove the threat from their systems.
\\
Diffie-Hellman-Merkle \cite{diffie1976new} is a security protocol used for allowing two parties to establish a secure connection. The protocol was created in 1976 and is still widely used to this day. After over 30 years of widespread usage we could imagine that we have perfected the implemention, we would have perfected its implementation. This is unfortunately not the case. A paper published in May of 2015 showcased vulnerabilities in the real world implementations of the DHM protocol. The vulnerability named ``LogJam'' in the paper ``Imperfect Forward Secrecy: How Diffie-Hellman Fails in Practice'' \cite{logjam2015}. This paper explains what the security issues are, what types of systems are affected and what you can do to protect yourself. Had you written your own protocol, you would have to verify its integrity on your own.  

%https://weakdh.org/imperfect-forward-secrecy.pdf
%https://weakdh.org/?utm_content=buffer6ce9c&utm_medium=social&utm_source=twitter.com&utm_campaign=buffer
\iffalse
% Good comment!
Additionally, due
to a breakdown in communication between cryptographers
and system implementers, there is evidence that suggests
the way we are using Diffie-Hellman in today’s protocols is
insufficient to protect against state-level actors
\fi
\subsection{Embedded Domain Specific Language}
A domain specific language (DSL) \cite{fowler2010domain} is a computer language that solves the problem of a specific domain. The DSL language does not have its own complier, but is instead converted to a different language which does have a compiler. The DSL source files are parsed into an abstract syntax tree(AST) and then outputted as a general purpose language. Languages such as Java, C++ or C\# are general purpose languages. What these types of languages have in common is that they provide a foundation for solving multiple domains. DSLs however are targeted at specific domains, and can often only be used for their intended purpose. Two popular examples of DSLs are Structured Query Language (SQL) and HyperText Markup Language (HTML) \cite{mernik2005and}.
\subsubsection{Benefits and Disadvantages}
The main benefit of using a DSL is its ability to provide a written language that is equivalent to the spoken language. The DSL can then become self-documenting as it should be easy to extract the purpose of a block of code.
\\\\
The disadvantage is that it takes a considerable amount of time and effort to create a DSL. One must define a grammar for the language, build an AST, then finally convert the AST to the source code for a compilable language. Another significant disadvantage is that this DSL is a new language. New developers are unlikely to have any previous experience with this language and there are no integrated development environments. The DSL will also add an extra step to debugging. There may be issues involved with the actual conversion from the DSL to the compiled language. These are generally errors that are hard to detect. The error may be in the actual usage or understanding of the DSL, performing actions that were unintended. This bring us to our final disadvantage, designing for change. A DSL is created to solve a specific task. If that task changes over time, the DSL must be updated or in the worst case, rewritten. 
\\\\
%we will be able to define a protocol and specify its rules in a coherent and self-explanatory fashion. 
An Embedded DSL does not have all these disadvantages however. The word ``embedded'' means that we are writing our DSL inside a host language. The DSL we write is then valid syntax for the host language. This removes a lot of the disadvantages with writing DSL, but severely limits us in our ability to freely express ourselves. This limitation is entirely dependant of the host language, some may be more limiting then others.
\\
We will create an eDSL to attempt to alleviate the complexity that developers face during implementation of network protocols. Our eDSL will try to enable a work flow guided by encapsulation. It will achieve this property by decomposing the complexity of the overall protocol into smaller elements. Ideally these elements will only have a single task to perform and will abide by the following rule: one input, one output. 



\iffalse

More background
Objectives
Why we need network protocols
Motivation
Explain DiffieHellman
\fi

% Can I change the wording of the objective to better match my actual implementation? 

\subsection{Objectives}
\label{sec:objectives}
\subsubsection{Primary}
\begin{enumerate}
    \item Write a library that utilises the Diffie-Hellman-Merkel key exchange for encrypting data 
over a public network.
        \begin{itemize}
            \item Initiate the establishment of a secure channel between two parties 
            \item Accept incoming connections for establishing a secure communication channel. 
            \item Generate a secure public key 
            \item Generate a secret \& secure personal key 
            \item Generate a secret shared key 
            \item Encrypt messages sent \& decrypt messages received
        \end{itemize}
    \item Create a domain specific language for usage of this library
        \begin{itemize}
            \item Make human readable
        \end{itemize}
\end{enumerate}
\subsubsection{Secondary}
\begin{enumerate}
    \item Make the library concurrent
        \begin{itemize}
            \item Allow multiple connections to be established simultaneously
        \end{itemize}
    \item Expand DSL to be capable of implementing additional protocols
        \begin{itemize}
            \item Needham-Schroeder protocol 
        \end{itemize}
\end{enumerate}

\subsubsection{Tertiary}
\begin{enumerate}
    \item Formally verify correctness / model checker
    \item Create State diagram showing state transitions
    \item Create Time-sequence diagram describing communication
\end{enumerate}


\subsection{Report Structure}
\begin{description}
  \item[Introduction] - Provides a high level view of the description of the problem.
  \item[Related Work] - Information regarding related work to this topic
  \item[Background] - Useful background information the reader should know
  \item[Tools] - The tools we decided to use to create our DSL
  \item[Initial Architecture] - Describes a birds eye view of out initial solution
  \item[Evaluation] - Critical and objective evaluation of our work 
  \item[Conclusion] - Summarizes our paper and proposes future work.
\end{description}


%Outline the structure of the report summarizing each chapter in one sentence.






